\documentclass[]{article}

% Packages
\usepackage[a4paper, total={6in, 10in}]{geometry}
\usepackage[normal, ruled]{caption}
\usepackage[margin=0pt, skip=6pt]{subcaption}
\usepackage{graphicx}
\usepackage{listings}
\usepackage{color}
\usepackage{xcolor}
\usepackage{siunitx}
\usepackage{hyperref}
\usepackage{booktabs}
\usepackage{amsmath}
\usepackage{multirow}
%\usepackage[bahasa]{babel}  % use this package to implement Indonesian lang

% Macros
\providecommand{\keywords}[1]{\textbf{\textit{Keywords---}} #1}
\newcolumntype{L}[1]{>{\raggedright\let\newline\\\arraybackslash\hspace{0pt}}m{#1}}
\newcolumntype{C}[1]{>{\centering\let\newline\\\arraybackslash\hspace{0pt}}m{#1}}
\newcolumntype{R}[1]{>{\raggedleft\let\newline\\\arraybackslash\hspace{0pt}}m{#1}}


% Settings
%% Hyperlink
\hypersetup{colorlinks,
  linkcolor={red!50!black},
  citecolor={blue!50!black},
  urlcolor={blue!80!black}
}

%% Figure
\graphicspath{{pics/}}

%% Listing Style
\lstdefinestyle{c}{
	belowcaptionskip=1\baselineskip,
	breaklines=true,
	frame=L,
	xleftmargin=\parindent,
	language=C,
	showstringspaces=false,
	basicstyle=\footnotesize\ttfamily,
	keywordstyle=\bfseries\color{green!40!black},
	commentstyle=\itshape\color{purple!40!black},
	identifierstyle=\color{blue},
	stringstyle=\color{orange},
	numberstyle=\tiny\color{gray}, % the style that is used for the line-numbers
	numbers=left,                  % where to put the line-numbers; possible values are (none, left, right)
	numbersep=7pt,                 % how far the line-numbers are from the code
	tabsize=2,	                   % sets default tabsize to 2 spaces
}

\lstdefinestyle{bash}{
	language=bash,
	backgroundcolor=\color{gray},
    basicstyle=\footnotesize\linespread{1.15}\ttfamily\color{white!20!black},
	identifierstyle=\color{white!20!black},
	keywordstyle=\bfseries\color{red!20!white},
	commentstyle=\color{blue!20!white},
	stringstyle=\color{orange},	
	belowcaptionskip=1\baselineskip,
	breaklines=true,
	frame=L,
    framesep=0.2cm,
    framerule=0pt,
	xleftmargin=\parindent,
	showstringspaces=false,
	numberstyle=\scriptsize\color{black}, % the style that is used for the line-numbers
	%numbers=left,                  % where to put the line-numbers; possible values are (none, left, right)
	numbersep=7pt,                 % how far the line-numbers are from the code
	tabsize=2,	                   % sets default tabsize to 2 spaces
}

% Header
\title{Report Title}
\author{Your Name Here}
\bibliographystyle{IEEEtran}

\begin{document}
\maketitle
\begin{abstract}
  Report abstract
\end{abstract}
\keywords{key1, key2, key3, key4}

\tableofcontents
\newpage

\section{Introduction}
bla\dots\dots\dots

\section{Section A}
bla\dots\dots\dots

\subsection{Subsection A.1}
bla\dots\dots\dots
\begin{enumerate}
  \item Jumble 1
    \href{http://tortoisehg.bitbucket.org/download/}{link}.
  \item Jumble 2
\end{enumerate}

\subsection{Subsection A.2}

\begin{lstlisting}[style=bash]
# Generated by TortoiseHg settings dialog

[ui]
username = user <user@hosting.com>
merge = kdiff3
\end{lstlisting}

\clearpage
\section{Menyiapkan Remote Repository}
\subsection{Membuat user baru melalui UI WDMyCloud}
\begin{enumerate}
  \item Buka UI WDMyCloud melalui web browser dengan memasukkan ip address
    WDMyCloud (192.168.1.134).
  \item Buka tab Users $\rightarrow$ klik add new user
    \begin{enumerate}
      \item Masukkan nama, username dan password
      \item Tambahkan full access ke folder Qimtronics dan Repo
    \end{enumerate}
  \item Buka tab Shares $\rightarrow$ Pilih folder user
    \begin{enumerate}
      \item Matikan Public Access
      \item Berikan Full Access user ke folder milik user
    \end{enumerate}
\end{enumerate}

\subsection{Membuat Remote Repository}
TODO: hg init di folder yang diinginkan.

\end{document}
